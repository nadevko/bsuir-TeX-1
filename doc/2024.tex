\documentclass{bsuir}

% Define the ENSTD PDF URL
\newcommand{\ENSTDPDFURL}{https://www.bsuir.by/m/12_100229_1_185586.pdf}

% Define ref_enstd command (using hyperref, loaded by bsuir.cls)
\newcommand{\refenstd}[1]{%
  \href{\ENSTDPDFURL\##1}{[#1]}%
}

% Define ref_enstd_text command
\newcommand{\refenstdtext}[2]{%
  #2~\refenstd{#1}%
}

% Define footnote_mark command (renamed to avoid conflict)
\newcommand{\footnotemarkcustom}[1]{%
  \textsuperscript{\tiny #1}%
}

% Define warning environment (simple, without tcolorbox)
\newenvironment{warning}[1]{%
  \vspace{0.5\baselineskip}
  \noindent\rule{2pt}{\baselineskip}\hspace{0.5em}%
  \begin{minipage}[t]{\dimexpr\textwidth-2pt-0.5em\relax}
  \textbf{Важно:} #1
  \end{minipage}
  \vspace{0.5\baselineskip}
}{}

% Define metrics environment (simple, without colors)
\newenvironment{metrics}{%
  \itshape📏 %
}{\normalfont}

% Document settings for cheat sheet
% Note: bsuir.cls sets 14pt, but we want 10pt for cheat sheet
% We'll override some settings
\makeatletter
\renewcommand{\normalsize}{%
  \@setfontsize\normalsize{10}{12}%
  \abovedisplayskip 10\p@ \@plus2\p@ \@minus5\p@
  \abovedisplayshortskip \z@ \@plus3\p@
  \belowdisplayshortskip 6\p@ \@plus3\p@ \@minus3\p@
  \belowdisplayskip \abovedisplayskip
  \let\@listi\@listI
}
\makeatother

% Page margins for cheat sheet
\geometry{
  left=1.5cm,
  right=1.5cm,
  top=1.5cm,
  bottom=1.5cm
}

% Heading numbering
\setcounter{secnumdepth}{3}

% Custom heading spacing
\titleformat{\section}{\normalfont\bfseries}{\thesection}{1em}{}
\titlespacing*{\section}{0pt}{1.2em}{0.8em}

\begin{document}
\normalsize

\chapter{Шпаргалка по СТП: Хронология и Оформление}

\section{Часть 1. Хронология и Процессы}

\footnotemarkcustom{1} Под термином ``диплом'' понимается как дипломный проект, так и дипломная работа \refenstd{1.1.1}.

\subsection{Подготовка и выбор темы}

\textbf{Сроки:}
\begin{itemize}
\item Темы объявляются кафедрой не позднее чем за \textbf{4 недели} до начала практики \refenstd{1.1.6}.
\item Заявление студента — не позднее чем за \textbf{3 недели} до практики.
\item Приказ ректора — через \textbf{10 дней} после начала практики \refenstd{1.1.7}.
\end{itemize}

\textbf{Изменения:}
\begin{itemize}
\item Уточнить название темы можно только в \textbf{первую неделю} практики \refenstd{1.1.6}.
\item Смена руководителя — только по уважительной причине (документально) до конца практики \refenstd{1.1.7}.
\end{itemize}

\subsection{Преддипломная практика}

\textbf{Задание:} Выдается в \textbf{первую неделю}. Должно быть подписано студентом и утверждено зав. кафедрой \refenstd{1.1.10}. Один экземпляр в РПЗ, второй на кафедре.

\textbf{Контроль:}
\begin{itemize}
\item Еженедельные опроцентовки.
\item \textbf{Технологический контроль} (нормоконтроль) осуществляет консультант от кафедры \refenstd{1.3.3}.
\end{itemize}

\subsection{Допуск и Защита}

\textbf{Рабочая комиссия:}
\begin{itemize}
\item Заседает не позднее чем за \textbf{2 недели} до ГЭК \refenstd{1.4.2}.
\item \textbf{Важно:} Комиссия не смотрит проект, если консультант не подписал свой раздел \refenstd{1.4.3}.
\item Решения: ``Одобрить'', ``Доработать'' (срок 1 неделя), ``Не готов''.
\end{itemize}

\textbf{Рецензирование:}
\begin{itemize}
\item Рецензент назначается за \textbf{1 месяц}.
\item Студент получает рецензию за \textbf{1 сутки} до защиты \refenstd{1.4.8}.
\item Исправлять проект по замечаниям рецензента \textbf{запрещено}.
\end{itemize}

\textbf{Защита:}
\begin{itemize}
\item Доклад до 15 минут.
\item Вопросы задают только члены ГЭК \refenstd{1.5.6}.
\item Оценка объявляется в тот же день.
\end{itemize}

\vspace{1em}
\hrule
\vspace{1em}

\section{Часть 2. Техническое оформление (Раздел 2 + Приложения)}

\subsection{2.1. Параметры страницы \refenstd{2.1.1} \refenstd{Л}}

\textbf{Поля и макет:}
\begin{itemize}
\item Формат: А4 (297×210 мм).
\item Левое: \textbf{30 мм}. Правое: \textbf{15 мм}. Верхнее/Нижнее: \textbf{20 мм}.
\item Текст только на \textbf{одной} стороне листа \refenstd{2.1.2}.
\end{itemize}

\textbf{Текст:}
\begin{itemize}
\item Шрифт: Times New Roman, \textbf{14 пт}.
\item Интервал: \textbf{1,0} (18 пт).
\item Абзацный отступ: \textbf{1,25 см} (12,5 мм) \refenstd{Л}.
\item Выравнивание: \textbf{По ширине}.
\item Переносы: Автоматические разрешены (в заголовках — \textbf{запрещены}).
\end{itemize}

\textbf{Нумерация:}
\begin{itemize}
\item Справа внизу, 10 мм от края листа.
\item Сквозная (включая Титульный, Задание, Приложения).
\item На Титульном, Реферате, Задании номер \textbf{не ставится}, но учитывается.
\end{itemize}

\textbf{Перенос заголовка:}
\begin{itemize}
\item Если текст после названия подраздела не помещается на текущей странице, то название переносится на следующую страницу \refenstd{2.1.1}.
\end{itemize}

\subsection{2.2. Заголовки \refenstd{2.2} \refenstd{Л}}

\textbf{Стили:}
\begin{itemize}
\item \textbf{РАЗДЕЛЫ (1):} ПРОПИСНЫЕ, \textbf{Полужирный}, С абзаца (1.25 см).
\item \textbf{Подразделы (1.1):} Строчные (с первой прописной), \textbf{Полужирный}, С абзаца.
\item \textbf{Ненумерованные} (ВВЕДЕНИЕ, ЗАКЛЮЧЕНИЕ, СПИСОК\ldots): ПРОПИСНЫЕ, \textbf{Полужирный}, \textbf{По центру}, Без абзаца.
\end{itemize}

\textbf{Правила верстки:}
\begin{itemize}
\item Точка в конце \textbf{не ставится}. Подчеркивания нет.
\item Если заголовок > 1 строки: вторая строка выравнивается \textbf{по началу текста первой} (с левого края верхней строки, а не с абзацного отступа) \refenstd{Л}.
\item Каждый \textbf{Раздел} начинается с новой страницы.
\end{itemize}

\textbf{Отступы (пустые строки):}
\begin{itemize}
\item Заголовок Раздела $\rightarrow$ Заголовок Подраздела: \textbf{1 строка}.
\item Заголовок $\rightarrow$ Текст: \textbf{1 строка}.
\item Текст $\rightarrow$ Новый заголовок: \textbf{1 строка}.
\item \textbf{Исключение:} Между пунктом и подпунктом строка нужна; между заголовком подпункта и текстом отступа нет \refenstd{Л}.
\item Между заголовками подпункта и пункта строка нужна (важен порядок) \refenstd{Л}.
\end{itemize}

\textbf{Пункты и подпункты:} \refenstd{2.2.3} \refenstd{2.2.4}
\begin{itemize}
\item Пункты нумеруют в пределах подраздела (или раздела, если подразделов нет).
\item Подпункты нумеруются в пределах каждого пункта.
\item Пункты, как правило, заголовков не имеют.
\item Номера разделов, подразделов и пунктов — \textbf{жирные}; номер подпункта — \textbf{не жирный} \refenstd{Л}.
\end{itemize}

\textbf{Дополнительные правила:}
\begin{itemize}
\item Если заголовок состоит из двух предложений, их разделяют точкой \refenstd{2.2.5}.
\item Между заголовками разделов и входящих в него подразделов допускается помещать небольшой вводный текст, предваряющий подраздел \refenstd{2.2.6}.
\item Продолжение заголовка идёт с левого края верхней строки, а не с абзацного отступа \refenstd{Л}.
\item Если заголовок раздела не помещается на странице, делаем page break \refenstd{Л}.
\item Пустые строки не стакаются (не накапливаются) \refenstd{Л}.
\end{itemize}

\textbf{СОДЕРЖАНИЕ:} \refenstd{2.2.7}
\begin{itemize}
\item Обязательный элемент пояснительной записки.
\item Помещают за заданием на проектирование.
\item Слово \texttt{СОДЕРЖАНИЕ}: ПРОПИСНЫЕ, полужирный, 14–16 пт, по центру.
\item Между словом и содержанием — пробельная строка.
\item Заголовки выравнивают, соподчиняя по разделам, подразделам и пунктам, смещая вертикали вправо на 2 знака.
\item Каждый заголовок соединяют отточием с номером страницы справа.
\end{itemize}

\subsection{2.3. Текст и Списки \refenstd{2.3}}

\textbf{Стиль изложения:} \refenstd{2.3.1}
\begin{itemize}
\item Текст должен быть четким и логично изложенным.
\item Обязательные требования: ``должен'', ``следует'', ``необходимо'', ``требуется, чтобы'', ``не допускается'', ``запрещается''.
\item Другие положения: ``допускают'', ``указывают'', ``применяют''.
\item Применять научно-технические термины по стандартам.
\item Запрещается применять иностранные термины при наличии равнозначных русских.
\end{itemize}

\textbf{Орфография и пунктуация:} \refenstd{2.3.2}
\begin{itemize}
\item Соблюдение правил орфографии и пунктуации.
\item Внимание на абзацы, перечисления, употребление чисел, символов и размерностей.
\end{itemize}

\textbf{Абзацы:} \refenstd{2.3.3}
\begin{itemize}
\item Небольшие по объему обособленные по смыслу части текста выделяют абзацами.
\end{itemize}

\textbf{Числа:} \refenstd{2.3.12}
\begin{itemize}
\item 1\ldots9 (без размерности) $\rightarrow$ словами (``пять опытов'').
\item >9 или с размерностью $\rightarrow$ цифрами (``10 опытов'', ``5 Вт'').
\item Диапазоны: ``от 2 до 5 км'' (размерность только в конце). ``Не более'', ``Не менее''.
\item Дробные числа в виде десятичных дробей.
\item Перед числами с размерностями не рекомендуется ``в'' или тире.
\item Порядковые числительные: цифрами с наращением падежного окончания (``2-м'', ``1-го'').
\item Количественные до десяти без единиц $\rightarrow$ словами (``на шести листах'').
\end{itemize}

\textbf{Списки (Перечисления):} \refenstd{2.3.4} \refenstd{2.3.5} \refenstd{2.3.6} \refenstd{2.3.7}
\begin{enumerate}
\item \textbf{Простое (тире):} С абзаца. В конце строки — \textbf{точка с запятой}. В конце списка — точка.
\item \textbf{Простое в подбор:} Допускается писать в подбор с текстом, отделяя запятой \refenstd{2.3.6}.
\item \textbf{Сложное (1, 2\ldots):} Если элементы — это предложения. Нумерация арабскими цифрами (без скобки). С прописной. В конце — \textbf{точка}.
\item \textbf{Детализация:} При дальнейшей детализации — арабские цифры со скобкой, с абзацного отступа соответствующего уровня \refenstd{2.3.8}.
\item \textbf{Ссылки:} Элементы обозначаются \texttt{а)}, \texttt{б)}. В тексте ссылка: ``см. \textbf{пункт} 1.7, б'' (без скобки, слово ``пункт'' или ``подпункт'' без сокращения) \refenstd{2.3.9}.
\item \textbf{Вводная фраза:} Все элементы должны подчиняться вводной фразе. Не обрывать на предлогах/союзах ``из'', ``на'', ``то'', ``как'' \refenstd{2.3.10}.
\end{enumerate}

\textbf{Математические знаки в тексте:} \refenstd{2.3.11}
\begin{itemize}
\item В тексте (кроме формул, таблиц и рисунков) писать словами:
\item Знак ``–'' минус перед отрицательными значениями.
\item Знаки > (больше), < (меньше), = (равно).
\item Знаки No (номер), \% (процент), Ø (диаметр), sin (синус), cos (косинус) без числовых значений.
\end{itemize}

\textbf{Единицы физических величин:} \refenstd{2.3.13} \refenstd{Т} \refenstd{У}
\begin{itemize}
\item В соответствии с ГОСТ 8.417–2002.
\item Буквенные обозначения величин и размерностей — в приложении Т.
\item Коэффициенты перевода — в приложении У.
\end{itemize}

\textbf{Обозначения единиц в формулах:} \refenstd{2.3.14}
\begin{itemize}
\item Не помещать обозначения единиц в одной строке с формулами в буквенной форме.
\item При подстановке числовых значений: обозначение единицы за результатом с пробелом (1 знак).
\end{itemize}

\textbf{Индексы и Символы (Критично!):} \refenstd{2.3.15} \refenstd{У}

\begin{warning}{Нормоконтроль часто возвращает работу из-за курсива в индексах!}
\end{warning}

\textbf{Правила начертания:}
\begin{itemize}
\item \textbf{Латиница} ($A, x, \max$) $\rightarrow$ \textbf{Курсив}.
\item \textbf{Русские} ($\text{вх}, \text{вых}, \text{дв}$) $\rightarrow$ \textbf{Прямой}.
\item \textbf{Греческие} ($\alpha, \omega, \pi$) $\rightarrow$ \textbf{Прямой}.
\item Образцы букв приведены в приложении Ф \refenstd{2.4.2}.
\end{itemize}

\textbf{Типы индексов:}
\begin{itemize}
\item Верхние: Цифры ($x^2$), штрих ($x'$), звездочка ($K^*$), буква T ($A^T$).
\item Нижние:
  \begin{itemize}
  \item Порядковые номера $\rightarrow$ Цифры ($x_1$).
  \item Смысловые сокращения $\rightarrow$ Буквы ($v_x$).
  \item Русские сокращения: \textbf{Прямой шрифт}, с точкой между сокращениями ($K_{\text{о.с.}}$, $U_{\text{вх}}$).
  \item Перечисление в индексе: через запятую ($\alpha_{(2,1)}$).
  \end{itemize}
\end{itemize}

\subsection{2.4. Формулы \refenstd{2.4} \refenstd{М}}

\textbf{Стиль изложения:} \refenstd{2.4.1}
\begin{itemize}
\item Не использовать: ``мы получили'', ``мы нашли'', ``определили'', ``получится'', ``выразится в виде'', ``будем иметь''.
\item Использовать: ``получаем'', ``определяем'', ``находим'', ``преобразуем к виду''.
\item Связующие слова (``следовательно'', ``откуда'', ``поскольку'', ``так как'', ``или'') — в начале строк, знаки препинания — непосредственно за формулой.
\item Если формулам предшествует фраза с обобщающим словом, после неё ставить двоеточие.
\item При ссылке на заимствованную формулу указывать ссылку на источник.
\end{itemize}

\textbf{Расположение и оформление:} \refenstd{2.4.2} \refenstd{2.4.3}
\begin{itemize}
\item По центру отдельной строки. Номер справа в круглых скобках \texttt{(2.1)}.
\item Знаки сложения, вычитания, корня, равенства размещать так, чтобы их середина была строго против горизонтальной черты дроби.
\item Минимум 1 пустая строка \textbf{ДО} и \textbf{ПОСЛЕ} формулы.
\end{itemize}

\textbf{Короткие формулы:} \refenstd{2.4.4}
\begin{itemize}
\item Однотипные формулы допускается располагать на одной строке, разделяя точкой с запятой.
\item Несложные и короткие формулы промежуточных выражений можно располагать в строке текста.
\end{itemize}

\textbf{Перенос:} \refenstd{2.4.5}
\begin{itemize}
\item Разрешен на знаках $+, -, \times$.
\item Знак \textbf{дублируется} в начале следующей строки.
\item На знаке деления, под корнем/интегралом — \textbf{запрещен}.
\end{itemize}

\textbf{Нумерация:} \refenstd{2.4.6}
\begin{itemize}
\item В пределах раздела: номер состоит из номера раздела и номера формулы, например, (2.7).
\item Если в разделе одна формула, её также нумеруют, например, (1.1).
\item Если формул не много, разрешается сквозная нумерация.
\item В приложениях: отдельная нумерация, например, (Б.2).
\end{itemize}

\textbf{Размещение номера:} \refenstd{2.4.7}
\begin{itemize}
\item При переносе части формулы номер располагают на последней строке.
\item Номер сложной формулы (дробь): середина номера на уровне черты дроби.
\item Ссылки в тексте: в круглых скобках с обязательным указанием слова ``формула'', ``уравнение'', ``выражение'', ``равенство'', ``передаточная функция'' и т.д.
\end{itemize}

\textbf{Экспликация (``где''):} \refenstd{2.4.7}
\begin{itemize}
\item С новой строки \textbf{без абзаца}. Двоеточие после ``где'' не ставить. Символы в столбик.
\item Альтернатива ``здесь'': после формулы точка, слово ``здесь'' с абзацного отступа с прописной.
\item Альтернатива с обобщающими словами: после них двоеточие, каждый символ с красной строки.
\end{itemize}

\begin{metrics}
Высота формул (включая отступы): \refenstd{М}
\begin{itemize}
\item Простая однострочная: $\sim$6 интервалов (24 мм).
\item С дробями/интегралами: $\sim$8 интервалов (32 мм).
\item В столбик с высокими знаками: 48 мм \refenstd{М}.
\end{itemize}
\end{metrics}

\subsection{2.5. Иллюстрации (Рисунки) \refenstd{2.5}}

\textbf{Виды иллюстраций:} \refenstd{2.5.1}
\begin{itemize}
\item Чертежи, схемы, графики, осциллограммы, цикло- и тактограммы, фотографии.
\item Все иллюстрации независимо от вида называются ``Рисунок''.
\end{itemize}

\textbf{Расположение:} \refenstd{2.5.2} \refenstd{2.5.3}
\begin{itemize}
\item По возможности ближе к разъясняющей части текста.
\item Допускается располагать в конце пояснительной записки в виде приложения.
\item Рекомендуемые размеры: приблизительно 92 × 150 мм и 150 × 240 мм.
\item Располагать после абзаца, в котором дана первая ссылка.
\item Можно размещать на отдельном листе несколько рисунков (лист за страницей с ссылкой на последний).
\item Между абзацами: по центру, одна пробельная строка до и после \refenstd{Н}.
\item Поворот только на 90° по часовой стрелке \refenstd{2.5.4}.
\end{itemize}

\textbf{Наименование и нумерация:} \refenstd{2.5.5}
\begin{itemize}
\item Всегда ``Рисунок''. Сквозная нумерация (Рисунок 5) или по разделам (Рисунок 2.1).
\item В приложениях: ``Рисунок А.2'' (или ``Рисунок А.1'' если одна иллюстрация).
\end{itemize}

\textbf{Подпись:} \refenstd{2.5.5}
\begin{itemize}
\item Под рисунком, по центру, без абзацного отступа. Без точки в конце.
\item Формат: \texttt{Рисунок [Номер] – [Название]}.
\item Рисунок и подпись выравниваются по центру страницы \refenstd{Л}.
\end{itemize}

\textbf{Легенда и расшифровка:} \refenstd{2.5.5}
\begin{itemize}
\item Пояснения (1 - корпус; 2 - вал) помещаются \textbf{перед} названием рисунка.
\item Расшифровки пишут в подбор, отделяя точкой с запятой.
\item Цифры, буквы, условные обозначения позиций — без скобок, отделяя от расшифровок тире.
\item Длина строк с пояснениями не должна выходить за границы рисунка.
\item Стандартные буквенные позиционные обозначения не расшифровывают.
\item Если обозначения разъясняются в тексте, расшифровки в подписи не допускаются.
\item Не разрешается часть деталей пояснять в тексте, а другую — в подписи.
\end{itemize}

\textbf{Отступы:} \refenstd{Л}
\begin{itemize}
\item 1 строка до картинки, 1 строка до подписи, 1 строка после подписи.
\item Пустые строки не стакаются \refenstd{Л}.
\end{itemize}

\textbf{Многостраничность:} \refenstd{2.5.6}
\begin{itemize}
\item На следующих листах писать \texttt{Рисунок [Номер], лист 2}.
\end{itemize}

\textbf{Ссылки:} \refenstd{2.5.6}
\begin{itemize}
\item Ссылки на все иллюстрации обязательны.
\item Рекомендуемые обороты: ``в соответствии с рисунком 2'', ``на рисунке 5.1 изображены...''.
\end{itemize}

\textbf{Требования к выполнению:} \refenstd{2.5.7} \refenstd{2.5.8}
\begin{itemize}
\item В соответствии с требованиями ЕСКД, ЕСТД и ЕСПД.
\item При использовании чертежей по ЕСКД: исключить рамки, угловые штампы, спецификации; заменить элементы прямоугольниками из штрихпунктирных линий; сократить надписи.
\item Выполнение: компьютерная техника, шариковая ручка с темной пастой, или карандаш.
\item Надписи: стандартный шрифт, высота строчных букв не менее 2,5 мм.
\item Прописные буквы в подписях и обозначениях — на 1/3 крупнее строчных.
\end{itemize}

\subsection{2.6. Таблицы \refenstd{2.6}}

\textbf{Назначение:} \refenstd{2.6.1}
\begin{itemize}
\item Для упрощения изложения текста с большим фактическим материалом.
\item Оформляют: справочные сведения, значения функций, данные исследований, результаты моделирования.
\end{itemize}

\textbf{Расположение:} \refenstd{2.6.1} \refenstd{2.6.3}
\begin{itemize}
\item За абзацем с первой ссылкой или на следующей странице.
\item Допускается оформлять в виде приложения.
\item Таблицу вместе с заголовком отделяют от текста пробельной строкой.
\item Заголовок и саму таблицу пробельной строкой не разделяют.
\item До таблицы и после таблицы — пустые строки \refenstd{Л}.
\end{itemize}

\textbf{Заголовок:} \refenstd{2.6.2} \refenstd{Л}
\begin{itemize}
\item НАД таблицей, слева, на уровне левой границы таблицы.
\item Формат: \texttt{Таблица [Номер] – [Название]}.
\item Первая строка — выравнивание по левому краю, вторая строка — продолжение названия \refenstd{Л}.
\item В приложениях: ``Таблица Б.2''.
\end{itemize}

\textbf{Разрыв таблицы:} \refenstd{2.6.4}
\begin{itemize}
\item Нижнюю черту перед разрывом можно не рисовать.
\item На новой странице справа: \texttt{Продолжение таблицы [Номер]}.
\item Таблица начнётся не сразу, а с фразы ``Продолжение таблицы'' \refenstd{Л}.
\item Обязательно повторять головку таблицы (или номера граф).
\item Головку допускается заменять нумерацией граф (нумерацию помещают и в первой части после головки).
\end{itemize}

\textbf{Заголовки граф:} \refenstd{2.6.5}
\begin{itemize}
\item Рекомендуется параллельно строкам таблицы (при необходимости — перпендикулярно).
\item Заголовки граф и строки боковика — с прописной, подзаголовки — со строчной (если не имеют самостоятельного значения).
\item Все заголовки в именительном падеже единственного числа.
\item Слова полностью без сокращений (кроме установленных обозначений).
\item Запрещается размещать в ячейке головки два заголовка, разделенные косой линией.
\item Графа ``№ п/п'' — \textbf{ЗАПРЕЩЕНА}. При необходимости нумерации — в первой графе через пробел перед наименованием.
\end{itemize}

\textbf{Единицы физических величин:} \refenstd{2.6.6}
\begin{itemize}
\item В графе или строке боковика: после наименования показателя, отделяя запятой.
\item Допускается графа ``Обозначение единицы физической величины'', если большая часть наименований с размерностями.
\end{itemize}

\textbf{Примечания:} \refenstd{2.6.7}
\begin{itemize}
\item К большей части строк: отдельная графа ``Примечание''.
\item К отдельным строкам или графам: отдельная строка в конце таблицы над нижней ограничивающей чертой.
\end{itemize}

\textbf{Деление таблицы:} \refenstd{2.6.8}
\begin{itemize}
\item С небольшим количеством граф допускается делить на части, помещать рядом, разделяя двойной линией или линией удвоенной толщины.
\item Головку повторяют в каждой части.
\end{itemize}

\textbf{Заполнение:} \refenstd{2.6.9}
\begin{itemize}
\item Число знаков после запятой одинаковое для каждого столбца.
\item Числовые значения одной величины: разряды один под другим.
\item Различные величины: посередине ячейки.
\item Последовательные интервалы: ``От... до... включ.'', ``Св... до... включ.''.
\item Пустые ячейки запрещены (ставить тире).
\end{itemize}

\textbf{Пояснения:} \refenstd{2.6.10} \refenstd{2.6.11}
\begin{itemize}
\item Пояснительная записка должна содержать краткие пояснения к таблице в целом.
\item Небольшой цифровой материал нецелесообразно оформлять в виде таблицы — давать текстом в виде колонок.
\end{itemize}

\subsection{2.7. Приложения \refenstd{2.7}}

\textbf{Назначение:} \refenstd{2.7.1}
\begin{itemize}
\item Информация справочного или второстепенного значения, но необходимая для освещения темы.
\item Математические формулы, номограммы, вспомогательные вычисления, описания алгоритмов, технические характеристики, спецификации.
\item Допускается использовать отдельно изданные конструкторские документы.
\end{itemize}

\textbf{Оформление:} \refenstd{2.7.2} \refenstd{2.7.3}
\begin{itemize}
\item Каждое с новой страницы. Включаются в общую нумерацию страниц.
\item Обозначение: Заглавные русские буквы (А, Б, В\ldots), кроме Ё, З, Й, О, Ч, Ъ, Ы, Ь.
\item Если одно приложение, оно также обозначается ПРИЛОЖЕНИЕ А.
\item В тексте на все приложения должны быть ссылки.
\item Приложения располагают в порядке ссылок на них в тексте.
\item Формат заголовка:
  \begin{center}
    ПРИЛОЖЕНИЕ А\\
    (обязательное/рекомендуемое/справочное)\\
    \textbf{Заголовок}
  \end{center}
\item Иногда после заголовка делают обратную ссылку к основному тексту.
\end{itemize}

\subsection{2.8. Библиография \refenstd{2.8}}

\textbf{Расположение:} \refenstd{2.8.1}
\begin{itemize}
\item В конце пояснительной записки перед приложениями.
\item Заголовок: \texttt{СПИСОК ИСПОЛЬЗОВАННЫХ ИСТОЧНИКОВ}.
\item С новой страницы, по центру.
\end{itemize}

\textbf{Ссылки в тексте:} \refenstd{2.8.2}
\begin{itemize}
\item Арабские цифры в квадратных скобках в возрастающем порядке.
\item Точка в предложении ставится \textbf{после} скобок: ``\ldots текст [1].''
\item Перенос только ссылки без предшествующего слова на новую строку не допускается.
\end{itemize}

\textbf{Сортировка:} \refenstd{2.8.3} \refenstd{2.8.4}
\begin{itemize}
\item В порядке упоминания в тексте.
\item Учебные, учебно-методические материалы и пособия — в конце списка (даже без ссылок в тексте).
\end{itemize}

\textbf{Оформление:} \refenstd{2.8.5} \refenstd{2.8.6}
\begin{itemize}
\item В соответствии с ГОСТ 7.1–2003.
\item Образцы описания различных типов источников приведены в требованиях.
\end{itemize}

\subsection{2.9. Сноски, примечания и примеры \refenstd{2.9}}

\textbf{Сноски:} \refenstd{2.9.1} \refenstd{2.9.2} \refenstd{Л}
\begin{itemize}
\item Знаки сноски: арабские цифры со скобкой, справа на уровне верхнего обреза слова, числа, символа, предложения.
\item Внизу страницы: тот же знак под короткой чертой перед текстом пояснения, с абзацного отступа.
\item В пояснительной сноске каждая запись стартует с абзацного отступа, номер сноски сверху с круглой скобкой \refenstd{Л}.
\end{itemize}

\textbf{Примечания:} \refenstd{2.9.3} \refenstd{Л}
\begin{itemize}
\item Размещают после текстового, графического или табличного материала.
\item Слово ``Примечание'' — с прописной буквы, с абзаца.
\item Если примечание одно: после слова ``Примечание'' тире, текст с прописной.
\item Если примечаний несколько: нумерация по порядку арабскими цифрами.
\item Примечание к таблице: в конце таблицы над нижней ограничивающей чертой.
\end{itemize}

\end{document}

