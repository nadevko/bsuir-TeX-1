\documentclass[std=01-2024]{bsuir}

% Подключаем библиографию (создадим временный файл для примера)
\usepackage{filecontents}
\begin{filecontents}{refs.bib}
@book{knuth,
  author = {Knuth, Donald E.},
  title = {The Art of Computer Programming},
  year = {1984},
  publisher = {Addison-Wesley},
  langid = {english}
}
@article{gost,
  title = {Оформление дипломных проектов},
  author = {Иванов, И. И.},
  journal = {Вестник БГУИР},
  year = {2024},
  langid = {russian}
}
\end{filecontents}

\addbibresource{refs.bib}

\begin{document}

% --- Титульный лист (заглушка) ---
% В реальности лучше вставлять готовый PDF:
% \insertpdf{titlepage.pdf}
\thispagestyle{empty}
\begin{center}
    \vspace*{5cm}
    \Huge \bfseries ПОЯСНИТЕЛЬНАЯ ЗАПИСКА
    
    \vspace{2cm}
    \large к дипломному проекту
\end{center}
\clearpage

% --- Содержание ---
\tableofcontents

% --- Введение ---
\anonsection{Введение}
Текст введения. Здесь описывается актуальность, цель и задачи.
Отступ абзаца равен 1.25 см. Шрифт Times New Roman 14pt.

% --- Глава 1 ---
\chapter{Обзор предметной области}
\section{Анализ существующих решений}

Пример ссылки на источник \cite{knuth}. Текст выравнивается по ширине.

\subsection{Подраздел с перечислением}
Пример маркированного списка (дефис вместо точки):
\begin{itemize}
    \item первый элемент списка;
    \item второй элемент списка;
    \item третий элемент списка.
\end{itemize}

Пример нумерованного списка:
\begin{enumerate}
    \item Первый пункт.
    \item Второй пункт.
\end{enumerate}

\section{Пример формул}
Формулы должны быть отцентрированы и иметь номер справа в скобках.

\begin{equation}
    E = mc^2
    \label{eq:einstein}
\end{equation}

\begin{where}
    \item[E] энергия системы;
    \item[m] масса;
    \item[c] скорость света.
\end{where}

% --- Глава 2 ---
\chapter{Разработка системы}

\section{Таблицы}
Пример оформления таблицы согласно STP.

\begin{gosttable}{
    colspec = {X[1,l] X[3,l]}, 
    caption = {Сравнение аналогов}
}
    Критерий & Описание \\
    Скорость & Высокая скорость обработки данных \\
    Цена & Условно-бесплатное ПО \\
\end{gosttable}

\section{Иллюстрации}
Пример вставки рисунка (требуется файл, используем заглушку).

\begin{figure}[htbp]
    \centering
    % \includegraphics[width=0.5\textwidth]{image.png} 
    \rule{10cm}{5cm} % Черный прямоугольник вместо картинки
    \caption{Схема архитектуры}
    \label{fig:arch}
\end{figure}

\section{Листинги кода}
Пример вставки кода (Python).

\begin{code}[lst:py]{Пример функции на Python}
def hello_world():
    print("Hello, BSUIR!")
    return True
\end{code}

% --- Заключение ---
\anonsection{Заключение}
В ходе работы было сделано всё, что требовалось. Ссылка на ГОСТ \cite{gost}.

% --- Список источников ---
\printbibliography

% --- Приложения ---
\appendix
\chapter{Листинг основного модуля}
Текст приложения А.

\chapter{Акт внедрения}
Текст приложения Б.

\end{document}